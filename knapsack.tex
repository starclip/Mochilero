\documentclass[10pt,letterpaper]{article}
\usepackage[utf8]{inputenc}
\usepackage[spanish]{babel}
\usepackage{graphicx}
\usepackage{booktabs}
\usepackage{multicol}
\usepackage{multirow}
\usepackage{xspace}
\usepackage{color, colortbl}
\usepackage{underscore}
\usepackage{tabu}
\usepackage{url}
\usepackage{ragged2e}
\usepackage{verbatim}
\usepackage{mathdots} 
\usepackage{amsmath, amssymb, amsbsy, amsfonts} 
\usepackage[left=3.5cm,right=3.5cm,top=3.5cm,bottom=3.5cm]{geometry}
\setlength{\parskip}{\baselineskip}
\begin{document} 
    \begin{titlepage} 
    \newcommand{\HRule}{\rule{\linewidth}{0.5mm}} 
    \center   
    \textsc{\Huge Instituto Tecnológico de Costa Rica}\\[1.5cm] 
    \textsc{\normalsize PROYECTO DE INVESTIGACIÓN DE OPERACIONES}\\[0.5cm] 
    \textsc{\normalsize PROYECTO 1}\\[0.5cm] 
    \HRule\\[0.4cm] 
    {\huge\bfseries \vspace{1cm} KNAPSACK PROBLEM}\\[0.4cm] 
    \HRule\\[2cm] 
    \textbf{\Large Estudiantes}\\[0.5cm] 
        \begin{minipage}{0.4\textwidth} 
        \begin{flushleft} 
            \large 
            Jason Barrantes Arce 
            \textsc{2015048456} 
        \end{flushleft} 
    \end{minipage} 
    ~ 
    \begin{minipage}{0.4\textwidth} 
        \begin{flushright} 
           	\large 
            Steven Bonilla Zúñiga 
            \textsc{2015056296} 
        \end{flushright} 
    \end{minipage} 
   \newline \newline 
   \newline  
   \textbf{\Large Profesor}\\[0.5cm] 
    \textsc{\normalsize Francisco Torres Rojas}\\[0.5cm] 
    \end{titlepage} 
    
\titlepage{\textbf{Modo Ejemplo:}} \newline \newline 
        Se resolverá un problema general por medio de diversos algoritmos que nos 
        permitan encontrar una solución a ese problema. 
        El problema que se nos plantea es sobre mochila. \\ 
        Hay que llevar una mochila con capacidad de 15 (kilos, gramos) para un viaje. 
        Tenemos una serie de objetos que podemos llevar, pero esos objetos tienen un respectivo peso 
        y valor que será producido de manera aleatoria. \ \ \newline \newline 
        Restricciones: 
        \begin{itemize} 
        \item \textbf{Capacidad:} La mochila tendrá una capacidad de 15 
        \item \textbf{Objetos:} Se generarán aleatoriamente 6 objetos. 
        \item \textbf{C{i}:} Varía entre $0 < C{i} \leq 7$. 
        \item \textbf{V{i}:} Varía entre $0 < V{i} \leq 20$. 
        \end{itemize} 
        Los tres algoritmos que vamos a implementar son: 
        \begin{itemize} 
        \item \textbf{Algoritmo de Programación Dinámica:} Algoritmo para el caso 0/1. 
        \item \textbf{Algoritmo Greedy Básico:} Cada vez se escoge el objeto más 
         valioso que quepa en lo que sobre de la mochila. 
        \item \textbf{Algoritmo Greedy Proporcional:} En este se calcula el rendimiento definido 
        como el valor del objeto dividido entre la capacidad que consume. Se escoge el de mayor rendimiento 
        \end{itemize} 
        \ En el caso de programación dinámica ya que nuestro objetivo es maximizar el valor que obtenemos, usamos la fórmula: 
        \[ \textsc{\normalsize MAX(Z)}\\[0.5cm] = \sum_{i=1}^{n}x_{i}v_{i} \] 
        \ Que está sujeto a:  
        \[ \sum x_{i}c_{i} \leq C \] 
        \ Con cada $x_{i}$ = 0 o 1.  
        \ \ \newline \newline 
        
Se muestra a continuación la tabla de objetos con su respectivo costo (peso) y valor 
        que fueron asignados aleatoriamente cumpliendo con las restricciones: 
\definecolor{Gray}{gray}{0.9}
\definecolor{LightCyan}{rgb}{0.88,1,1}
\begin{center}
\begin{tabu} to 0.6\textwidth { | X[l] | X[l] | X[l] | } 
\hline
\rowcolor{Gray}
\textbf{Nombre} & \textbf{Costo} & \textbf{Valor}\\
\hline
Object 1 & 5 & 16 \\
\hline
Object 2 & 3 & 3 \\
\hline
Object 3 & 2 & 1 \\
\hline
Object 4 & 3 & 16 \\
\hline
Object 5 & 3 & 13 \\
\hline
Object 6 & 5 & 7 \\
\hline
\end{tabu} \\
\end{center}
\section{Algoritmo Dinámico $0/1$} 
        Es un problema bidimensional, cuyo objetivo es maximizar la ganancia. 
        En nuestro caso, queremos maximizar la cantidad de valores obtenidos por los objetos. 
        Para solucionar el problema vamos a hacer uso de una tabla (m+1) x n, donde n es la cantidad 
        de objetos disponibles y m es la cantidad de espacio disponible de la mochila. 
        Como se menciona en las restricciones del problema $n = 6$ y $m = 15$, por lo que tendremos 
        una tabla (16x6). \newline \newline \newline 
        \textbf{\Large Fórmula Matemática} 
        \[ \textsc{\normalsize MAX(Z)}\\[0.5cm] = \sum_{i=1}^{6}x_{i}v_{i} \] 
        \ Sujeto a:  
        \[ \sum x_{i}c_{i} \leq 15 \] 
        \ En otras palabras tendremos:  
        
\[ \textsc{\normalsize Z}\\[0.5cm] = 16x_{1}+3x_{2}+1x_{3}+16x_{4}+13x_{5}+7x_{6} \]
Sujeto a: 
\[ 5x_{1}+3x_{2}+2x_{3}+3x_{4}+3x_{5}+5x_{6} \leq 15 \]
\newline Ahora proseguimos realizando la tabla dinámica.
\definecolor{Gray}{gray}{0.5}
\definecolor{GreenBlack}{RGB}{2,80,0}
\begin{center}
\begin{tabu} to 1.0\textwidth { | c | c | c | c | c | c | c | }
\hline
\cellcolor{Gray}\color{black}\textbf{X} & \cellcolor{Gray}\color{black}\textbf{1} & \cellcolor{Gray}\color{black}\textbf{2} & \cellcolor{Gray}\color{black}\textbf{3} & \cellcolor{Gray}\color{black}\textbf{4} & \cellcolor{Gray}\color{black}\textbf{5} & \cellcolor{Gray}\color{black}\textbf{6} \\ 
\hline
\cellcolor{Gray}\color{black}0 & \textcolor{red}{ 0} & \textcolor{red}{ 0} & \textcolor{red}{ 0} & \textcolor{red}{ 0} & \textcolor{red}{ 0} & \textcolor{red}{ 0} \\ 
\hline
\cellcolor{Gray}\color{black}1 & \textcolor{red}{ 0} & \textcolor{red}{ 0} & \textcolor{red}{ 0} & \textcolor{red}{ 0} & \textcolor{red}{ 0} & \textcolor{red}{ 0} \\ 
\hline
\cellcolor{Gray}\color{black}2 & \textcolor{red}{ 0} & \textcolor{red}{ 0} & \textcolor{GreenBlack}{ 1} & \textcolor{red}{ 1} & \textcolor{red}{ 1} & \textcolor{red}{ 1} \\ 
\hline
\cellcolor{Gray}\color{black}3 & \textcolor{red}{ 0} & \textcolor{GreenBlack}{ 3} & \textcolor{red}{ 3} & \textcolor{GreenBlack}{ 16} & \textcolor{red}{ 16} & \textcolor{red}{ 16} \\ 
\hline
\cellcolor{Gray}\color{black}4 & \textcolor{red}{ 0} & \textcolor{GreenBlack}{ 3} & \textcolor{red}{ 3} & \textcolor{GreenBlack}{ 16} & \textcolor{red}{ 16} & \textcolor{red}{ 16} \\ 
\hline
\cellcolor{Gray}\color{black}5 & \textcolor{GreenBlack}{ 16} & \textcolor{red}{ 16} & \textcolor{red}{ 16} & \textcolor{GreenBlack}{ 17} & \textcolor{red}{ 17} & \textcolor{red}{ 17} \\ 
\hline
\cellcolor{Gray}\color{black}6 & \textcolor{GreenBlack}{ 16} & \textcolor{red}{ 16} & \textcolor{red}{ 16} & \textcolor{GreenBlack}{ 19} & \textcolor{GreenBlack}{ 29} & \textcolor{red}{ 29} \\ 
\hline
\cellcolor{Gray}\color{black}7 & \textcolor{GreenBlack}{ 16} & \textcolor{red}{ 16} & \textcolor{GreenBlack}{ 17} & \textcolor{GreenBlack}{ 19} & \textcolor{GreenBlack}{ 29} & \textcolor{red}{ 29} \\ 
\hline
\cellcolor{Gray}\color{black}8 & \textcolor{GreenBlack}{ 16} & \textcolor{GreenBlack}{ 19} & \textcolor{red}{ 19} & \textcolor{GreenBlack}{ 32} & \textcolor{red}{ 32} & \textcolor{red}{ 32} \\ 
\hline
\cellcolor{Gray}\color{black}9 & \textcolor{GreenBlack}{ 16} & \textcolor{GreenBlack}{ 19} & \textcolor{red}{ 19} & \textcolor{GreenBlack}{ 32} & \textcolor{red}{ 32} & \textcolor{red}{ 32} \\ 
\hline
\cellcolor{Gray}\color{black}10 & \textcolor{GreenBlack}{ 16} & \textcolor{GreenBlack}{ 19} & \textcolor{GreenBlack}{ 20} & \textcolor{GreenBlack}{ 33} & \textcolor{red}{ 33} & \textcolor{red}{ 33} \\ 
\hline
\cellcolor{Gray}\color{black}11 & \textcolor{GreenBlack}{ 16} & \textcolor{GreenBlack}{ 19} & \textcolor{GreenBlack}{ 20} & \textcolor{GreenBlack}{ 35} & \textcolor{GreenBlack}{ 45} & \textcolor{red}{ 45} \\ 
\hline
\cellcolor{Gray}\color{black}12 & \textcolor{GreenBlack}{ 16} & \textcolor{GreenBlack}{ 19} & \textcolor{GreenBlack}{ 20} & \textcolor{GreenBlack}{ 35} & \textcolor{GreenBlack}{ 45} & \textcolor{red}{ 45} \\ 
\hline
\cellcolor{Gray}\color{black}13 & \textcolor{GreenBlack}{ 16} & \textcolor{GreenBlack}{ 19} & \textcolor{GreenBlack}{ 20} & \textcolor{GreenBlack}{ 36} & \textcolor{GreenBlack}{ 46} & \textcolor{red}{ 46} \\ 
\hline
\cellcolor{Gray}\color{black}14 & \textcolor{GreenBlack}{ 16} & \textcolor{GreenBlack}{ 19} & \textcolor{GreenBlack}{ 20} & \textcolor{GreenBlack}{ 36} & \textcolor{GreenBlack}{ 48} & \textcolor{red}{ 48} \\ 
\hline
\cellcolor{Gray}\color{black}15 & \textcolor{GreenBlack}{ 16} & \textcolor{GreenBlack}{ 19} & \textcolor{GreenBlack}{ 20} & \textcolor{GreenBlack}{ 36} & \textcolor{GreenBlack}{ 48} & \textcolor{red}{ 48} \\ 
\hline
\end{tabu} \\
\end{center}
El resultado es el siguiente: 
Las soluciones de X son las siguientes: 
\[ \textsc{\normalsize Z}\\[0.5cm] = (16 * 1)+(3 * 1)+(1 * 0) + (16 * 1)+(13 * 1)+(7 * 0) \]
\[ \textsc{\normalsize Z}\\[0.5cm] = 48 \] 
\[ X_{1} = 1, X_{2} = 1, X_{3} = 0, X_{4} = 1, X_{5} = 1, X_{6} = 0 \]\newline Esta sujeto a: 
\[ (5 * 1)+(3 * 1)+(2 * 0) +(3 * 1)+(3 * 1)+(5 * 0)\leq 15 \]\newline El algoritmo tarda aproximadamente: 0.076000 segundos en ejecutarse
\section{Algoritmo Greedy Básico} 
        Es un algoritmo que soluciona problemas que a primera vista parece ser 
        óptimo. Es característico porque es muy sencillo de entender y explicar. 
        Se escogen los objetos más valiosos que entren en lo que sobra de la mochila. 
        \[ Obj_{i} = (Costo, Valor), i = 0...n \]
\[ Obj_{1} = (5, 16), Obj_{2} = (3, 3), Obj_{3} = (2, 1), Obj_{4} = (3, 16), Obj_{5} = (3, 13), Obj_{6} = (5, 7) \]
\newline Ahora proseguimos realizando la tabla greedy básico.
Se muestra a continuación la tabla greedy con los resultados: 
\definecolor{Gray}{gray}{0.9}
\definecolor{LightCyan}{rgb}{0.88,1,1}
\begin{center}
\begin{tabu} to 0.6\textwidth { | X[l] | X[l] | X[l] | } 
\hline
\rowcolor{Gray}
\textbf{Objeto} & \textbf{Peso} & \textbf{Valor}\\
\hline
Object 4 & 3 & 16 \\
\hline
Object 5 & 3 & 13 \\
\hline
Object 1 & 5 & 16 \\
\hline
Object 3 & 2 & 1 \\
\hline
\end{tabu} \\
\end{center}
\[ \textsc{\normalsize Z}\\[0.5cm] = 46 \] 
\newline El algoritmo tarda aproximadamente: 0.026000 segundos en ejecutarse
\section{Algoritmo Greedy Proporcional} 
        Es un algoritmo que soluciona problemas que a primera vista parece ser 
        óptimo. Es característico porque es muy sencillo de entender y explicar. 
        Se escogen los objetos más valiosos que entren en lo que sobra de la mochila, ya sea por su rendimiento. 
        \[ Obj_{i} = (Costo, Valor), i = 0...n \]
\[ Obj_{1} = (5, 16), Obj_{2} = (3, 3), Obj_{3} = (2, 1), Obj_{4} = (3, 16), Obj_{5} = (3, 13), Obj_{6} = (5, 7) \]
\newline Ahora proseguimos realizando la tabla greedy proporcional.
Se muestra a continuación la tabla greedy proporcional con los resultados: 
\definecolor{Gray}{gray}{0.9}
\definecolor{LightCyan}{rgb}{0.88,1,1}
\begin{center}
\begin{tabu} to 0.6\textwidth { | X[l] | X[l] | X[l] | X[l] |} 
\hline
\rowcolor{Gray}
\textbf{Objeto} & \textbf{Peso} & \textbf{Valor} & \textbf{Rend}\\
\hline
Object 4 & 3 & 16 & 5 \\
\hline
Object 1 & 5 & 16 & 3 \\
\hline
Object 5 & 3 & 13 & 4 \\
\hline
Object 2 & 3 & 3 & 1 \\
\hline
\end{tabu} \\
\end{center}
\[ \textsc{\normalsize Z}\\[0.5cm] = 48 \] 
\newline El algoritmo tarda aproximadamente: 0.022000 segundos en ejecutarse
\end{document}
